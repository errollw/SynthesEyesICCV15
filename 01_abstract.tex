%!TEX root = 00_main.tex
Images of the eye are used in several computer vision problems, such as facial feature localization and gaze estimation.
%, iris-biometrics, and deception analysis.
Large-scale supervised methods for these problems require time-consuming data collection procedures and unreliable manual annotation of training images.
%, which is error-prone and slows down progress in these areas.
Instead we propose synthesizing perfectly labelled photo-realistic training data in a fraction of the time.
We used computer graphics techniques to build a collection of dynamic eye-region models from head scan geometry.
%
These were randomly posed to synthesize close-up eye images with a wide range of head poses, gaze directions, and illumination conditions.
%
The resulting dataset (\dataset) was used to train two systems: an eye-region landmark detector and an appearance-based gaze estimatior.
% The model is able to simulate the large variability of real eyes, including pupil dilation, eyelid motion and corresponding changes in its shape, as well as iris colour variations.
%dynamic changes in shape
We demonstrate the benefits of synthesized training data by out-performing other state-of-the-art methods for eye-shape registration in the wild, and performing competitively in a challenging cross-dataset evaluation of gaze estimation.
% needs reword.
We also use our rendering framework to examine the effects of low illumination variance and less realism in training data.
%