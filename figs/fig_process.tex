%!TEX root = ../00_main.tex

\begin{figure*}
    \centering
    \begin{subfigure}[t]{0.195\textwidth}
        \includegraphics[width=\textwidth]{process_f02_01}
        \caption{}
        \label{fig:process_original_scan}
    \end{subfigure}
    \hfill
    \begin{subfigure}[t]{0.195\textwidth}
        \includegraphics[width=\textwidth]{process_f02_02}
        \caption{}
        \label{fig:process_retopo}
    \end{subfigure}
    \hfill
    \begin{subfigure}[t]{0.195\textwidth}
        \includegraphics[width=\textwidth]{process_f02_03}
        \caption{}
        \label{fig:process_displaced_subdiv}
    \end{subfigure}
    \hfill
    \begin{subfigure}[t]{0.195\textwidth}
        \includegraphics[width=\textwidth]{process_f02_04}
        \caption{}
        \label{fig:process_ldmks}
    \end{subfigure}
    \hfill
    \begin{subfigure}[t]{0.195\textwidth}
        \includegraphics[width=\textwidth]{process_f02_05}
        \caption{}
    \end{subfigure}
    \caption{Overview of the model preparation process. The original 3D head scans with 1.4 million polygons (a) are first retopologised to 9,000 polygons (b), surface details are stored with displacement maps (c), 3D iris and eyelid landmarks are annotated manually (d). Sample resulting rendering (e).}
    \label{fig:process}
\end{figure*}